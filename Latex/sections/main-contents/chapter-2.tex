\section{Các nghiên cứu liên quan}
\textbf{Linear Regression}: Trong nghiên cứu này, Arif Mudi Priyatno và những tác giả khác \cite{b1} (2023) đã so sánh hai phương pháp học máy Random Forest Regression và Linear Regression để dự báo cổ phiếu BBCA, nhằm đưa ra giải pháp chính xác và hiệu quả hơn cho quyết định đầu tư và giao dịch của nhà đầu tư. Kết quả đánh giá của các độ đo MSE, RMSE, MAE, MAPE đối với Linear Regression là 0,12848, 0,35807, 0,29570 và 0,0036\%, trong khi đối với Random Forest Regression là 27473,76, 158,04, 142,70 và 1,7153\%. Những kết quả này cho thấy Linear Regression hoạt động tốt hơn trong hiệu suất dự báo.
\par
\textbf{GRU}: Mochamad Ridwan, Kusman Sadik và Farit Mochamad Afendi \cite{b2} (2024) đã tiến hành đánh giá tính hiệu quả của hai mô hình ARIMA và GRU trong việc dữ báo giá cổ phiếu ở tần suất cao, đặc biệt là dữ liệu chứng khoán cấp độ phút từ ngân hàng HIMBARA. Hiệu suất dự đoán của mỗi phương pháp được đánh giá bằng cách sử dụng MAPE làm độ đo. Về độ chính xác dự báo, mô hình GRU vượt trội hơn mô hình ARIMA. Mô hình GRU đạt được MAPE là 0,77\% đối với cổ phiếu BMRI, trong khi mô hình ARIMA đạt được MAPE là 4,09\%. Mô hình GRU dự đoán MAPE là 0,34\% cho cổ phiếu BBRI, trong khi mô hình ARIMA dự đoán MAPE là 3,02\%. Đối với cổ phiếu BBNI, mô hình GRU đạt được MAPE là 0,63\%, trong khi mô hình ARIMA đạt được MAPE là 1,52\%. Mô hình GRU đạt được MAPE là 0,58\% cho cổ phiếu BBTN, trong khi mô hình ARIMA đạt được MAPE là 6,2\%. Nghiên cứu này bổ sung một góc nhìn mới cho cuộc thảo luận bằng cách so sánh hai phương pháp lập mô hình: mô hình ARIMA truyền thống và mô hình GRU học sâu phức tạp, cả hai đều được áp dụng cho dữ liệu tần số cao.
\par
\textbf{LSTM}: Shahzad Zaheer và những tác giả khác \cite{b3} (2023) đã sử dụng LSTM và những mô hình học sâu như RNN, CNN,… để lấy dữ liệu chứng khoán đầu vào và dự đoán hai thông số cổ phiếu là giá đóng và giá cao cho ngày hôm sau và kết quả cho thấy mô hình nào thực hiện tốt nhất.
\par
\textbf{ARIMA}: Prapanna Mondal, Labani Shit và Saptarsi Goswami \cite{b4} (2014) đã tiến hành một nghiên cứu trên 56 cổ phiếu từ các lĩnh vực khác nhau, độ chính xác của mô hình ARIMA trong dự đoán giá cổ phiếu cao hơn 85\%, từ đó cho thấy ARIMA mang lại độ chính xác cao về sự dự đoán.
\par
\textbf{Meta-Learning}: Cheng-Wen Hsu cùng các tác giả khác \cite{b5} đã nghiên cứu và đề xuất một khung học meta (meta-learning) dự đoán giá cổ phiếu trong ngắn hạn sử dụng mạng nơ-ron tích chập (CNN) bao gồm mạng nơ-ron tích chập thời gian, mạng nơ-ron tích chập hoàn toàn và mạng nơ-ron dư. Một phương pháp dán nhãn mới gọi là "slope-detection" được đề xuất, sử dụng khung thời gian trượt và các nhãn dự đoán bao gồm "tăng nhiều", "tăng", "giảm" và "giảm nhiều". Khung học meta được đánh giá trên S\&P500, cho thấy sự cải thiện đáng kể về độ chính xác dự đoán và lợi nhuận.
\par
\textbf{N-HiTS}: Cristian Challu và các tác giả khác đã tiến hành nghiên cứu và giới thiệu N-HiTS \cite{b6}, một mô hình giải quyết cả hai thách thức này bằng cách kết hợp các kỹ thuật phân cấp (hierarchical interpolation) và lấy mẫu dữ liệu đa tốc độ ( multi-rate data sampling). Các kỹ thuật này cho phép phương pháp đề xuất xây dựng dự báo theo trình tự, nhấn mạnh các thành phần với tần số và thang đo khác nhau trong khi phân tách tín hiệu đầu vào và tổng hợp dự báo. Họ chứng minh rằng kỹ thuật phân cấp có thể xấp xỉ hiệu quả các dự báo dài hạn tùy ý trong trường hợp có độ trơn (smoothness). Ngoài ra, họ tiến hành các thí nghiệm trên bộ dữ liệu quy mô lớn từ các tài liệu dự báo dài hạn, minh họa những ưu điểm của phương pháp N-HiTS so với các phương pháp tiên tiến khác. Cụ thể, N-HiTS cung cấp độ chính xác trung bình được cải thiện gần 20\% so với kiến trúc Transformer mới nhất trong khi giảm thời gian tính toán đáng kể (50 lần).