\section{Các nghiên cứu liên quan}
\textbf{Linear Regression}: Trong nghiên cứu này, Arif Mudi Priyatno và những tác giả khác \cite{lr} (2023) đã so sánh hai phương pháp học máy Random Forest Regression và Linear Regression để dự báo cổ phiếu BBCA. Kết quả đánh giá các độ đo MSE, RMSE, MAE, MAPE của Linear Regression thấp hơn nhiều so với Random Forest Regression. Cho thấy Linear Regression hoạt động tốt hơn trong hiệu suất dự báo.
Linear Regression có những hạn chế như khả năng chịu biến động thị trường, không đủ linh hoạt để nắm bắt các biến động đột ngột.
\par
\textbf{ARIMA}: Prapanna Mondal, Labani Shit và Saptarsi Goswami \cite{arima} (2014) đã tiến hành một nghiên cứu trên 56 cổ phiếu từ các lĩnh vực khác nhau, độ chính xác của mô hình ARIMA trong dự đoán giá cổ phiếu cao hơn 85\%, cho thấy ARIMA mang lại độ chính xác cao về sự dự đoán.
ARIMA có những hạn chế trong việc nắm bắt các mẫu dữ liệu phức tạp và phi tuyến tính hơn so với GRU.
\par
\textbf{RNN}: Hansika Hewamalage, Christoph Bergmeir và Kasun Bandara đã tiến hành một nghiên cứu về việc sử dụng Mạng Nơ-ron Tái Phát (RNN) \cite{rnn} trong dự báo chuỗi thời gian. Họ nhấn mạnh rằng RNN đã trở thành phương pháp dự báo cạnh tranh, nhưng vẫn cần cải thiện để đạt được tính tự động và hiệu quả. 
Dẫu vậy, RNN đòi hỏi nhiều tài nguyên tính toán và có khả năng gặp khó khăn trong việc mô hình hóa tính mùa vụ của các chuỗi thời gian.
\par
\textbf{LSTM}: Shahzad Zaheer và những tác giả khác \cite{lstm} (2023) sử dụng LSTM và những mô hình học sâu như RNN, CNN,… lấy dữ liệu chứng khoán đầu vào, dự đoán hai thông số cổ phiếu là giá đóng và giá cao cho ngày hôm sau.
LSTM chưa được tối ưu hiệu quả về hiệu suất hoặc cấu trúc một cách hiệu quả trong thí nghiệm này so với các phương pháp khác.
\par
\textbf{GRU}: Mochamad Ridwan, Kusman Sadik, và Farit Mochamad Afendi \cite{gru} (2024) đã so sánh hiệu suất của hai mô hình ARIMA và GRU trong dự báo giá cổ phiếu ở tần suất cao từ ngân hàng HIMBARA. Sử dụng MAPE làm độ đo, họ thấy rằng mô hình GRU vượt trội hơn mô hình ARIMA.
\par
\textbf{VARMA}: Gustavo Fruet Dias và George Kapetanios \cite{varma} đã nghiên cứu và đề xuất sử dụng các mô hình VARMA. Họ khắc phục những khó khăn tự nhiên trong việc ước tính các mô hình VARMA chiều trung bình và chiều cao bằng khung MLE bằng cách áp dụng công cụ ước tính IOLS. Ứng dụng thực nghiệm của họ cho thấy mô hình VARMA là lựa chọn thay thế khả thi khi dự báo với nhiều yếu tố dự báo. Cho thấy rằng các mô hình VARMA hoạt động tốt hơn các mô hình AR(1), ARMA(1,1), Bayesian VAR và các mô hình nhân tố, xem xét các kích thước mô hình khác nhau.
\par
\textbf{Kalman Filter}: Xu-Yan và Zhang Guosheng \cite{kalmanfilter} đã tiến hành nghiên cứu về Kalman Filter và MATLAB dựa trên sự biến động của thị trường chứng khoán và các tính năng theo dõi động của Kalman Filter. Kết quả của nghiên cứu cho thấy Kalman Filter trong việc dự đoán có hiệu quả, đơn giản và nhanh chóng.
Yêu cầu giả định về sự tuyến tính và phân phối Gaussian của nhiễu, không phù hợp cho các hệ thống phi tuyến và dữ liệu không tuân theo phân phối Gaussian.
\par
\textbf{Meta-Learning}: Cheng-Wen Hsu cùng các tác giả khác \cite{meta} đã nghiên cứu và đề xuất một khung học meta (meta-learning) dự đoán giá cổ phiếu trong ngắn hạn sử dụng mạng nơ-ron tích chập (CNN) bao gồm mạng nơ-ron tích chập thời gian, mạng nơ-ron tích chập hoàn toàn và mạng nơ-ron dư cho thấy sự cải thiện đáng kể về độ chính xác dự đoán và lợi nhuận.
Meta-learning được sử dụng để dự báo xu hướng giá cổ phiếu ngắn hạn, bên cạnh đó phương pháp gán nhãn "rise plus", "rise", "fall" và "fall plus" có thể bị hạn chế đối với các loại chứng khoản hoặc cổ phiếu khác.
\par
\textbf{NBEATS}: Boris N. Oreshkin, Nicolas Chapados, Dmitri Carpov và Yoshua Bengio đã đưa ra NBEATS \cite{nbeats}. Công trình này xây dựng trên nền tảng của nghiên cứu trước đó trong lĩnh vực dự báo chuỗi thời gian bằng deep learning và các phương pháp tạo ra mô hình dự báo có khả năng giải thích cao hơn. %Công trình này đề xuất một phương pháp mới, kết hợp các yếu tố của deep learning và các khối dư lượng để tạo ra một mô hình dự báo chuỗi thời gian có khả năng giải thích, áp dụng rộng rãi và nhanh chóng.
Mặc dù hiệu quả trong dự báo chuỗi thời gian, nhưng yêu cầu tài nguyên tính toán lớn và khả năng giải thích kết quả dự báo hạn chế.
\par
\textbf{N-HiTS}: Cristian Challu và các tác giả khác đã tiến hành nghiên cứu và giới thiệu N-HiTS \cite{nhits}, một mô hình giải quyết cả hai thách thức này bằng cách kết hợp các kỹ thuật phân cấp (hierarchical interpolation) và lấy mẫu dữ liệu đa tốc độ ( multi-rate data sampling). %Các kỹ thuật này cho phép phương pháp đề xuất xây dựng dự báo theo trình tự, nhấn mạnh các thành phần với tần số và thang đo khác nhau trong khi phân tách tín hiệu đầu vào và tổng hợp dự báo. Họ chứng minh rằng kỹ thuật phân cấp có thể xấp xỉ hiệu quả các dự báo dài hạn tùy ý trong trường hợp có độ trơn (smoothness). Ngoài ra, họ tiến hành các thí nghiệm trên bộ dữ liệu quy mô lớn từ các tài liệu dự báo dài hạn, minh họa những ưu điểm của phương pháp N-HiTS so với các phương pháp tiên tiến khác. Cụ thể, N-HiTS cung cấp độ chính xác trung bình được cải thiện gần 20\% so với kiến trúc Transformer mới nhất trong khi giảm thời gian tính toán đáng kể (50 lần). 
Tuy nhiên, N-HiTS khá phức tạp trong việc triển khai.